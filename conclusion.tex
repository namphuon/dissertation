\chapter{Conclusion and future work}\label{conclusion}
\index{Conclusion@\emph{Conclusion}}%

\section{Conclusion}
Sequence alignment is a vital step in many bioinformatics analyses.  From the alignment, the phylogenetic relationship between the different sequences in the alignment can be inferred.  Under the context of phylogenetic placement, I have shown that the standard approach of using a single HMM for aligning sequences to an existing alignment degrades when the sequences come from distantly related taxa, and that new methods are necessary for aligning evolutionarily divergent sequences.  I present such a technique in Chapter~\ref{hmmfamily}.  I describe the basic outline of the families of HMMs technique and how it can be used to align sequences to an existing backbone alignment.  

In Chapter~\ref{sepp_chapter}, I implemented the families of HMMs technique within SEPP and apply SEPP toward the phylogenetic placement problem.  I presented a simulation study and showed that SEPP resulted in better placement accuracy than HMMALIGN+pplacer and PaPaRa+pplacer on difficult datasets.  More importantly, I validated the hypothesis that using multiple HMMs can result in significantly better phylogenetic placement accuracy than using a single HMM.  This result forms the basis for the remaining chapters of my dissertation.

In Chapter~\ref{tipp_chapter}, I presented TIPP, an extension of SEPP by including statistical support measures, and showed its performance on  taxonomic identification and profiling.  I showed that using multiple HMMs resulted in better classification accuracy than using a single HMM.  Most interestingly, I showed that under the context of taxonomic identification, requiring a high statistical support threshold for classification resulted in the best overall results, however, under the context of taxonomic profiling, using the minimum support threshold resulted in the best overall taxonomic profiles.  Thus, the choice of the statistical support threshold depends on the application.

In Chapter~\ref{upp_chapter}, I presented UPP, a modification of SEPP for ``de novo'' sequence alignment.  SEPP requires a backbone alignment as input.  I presented a new technique to intelligently select the set of sequences to form the backbone alignment, and then applied the families of HMMs technique to complete the alignment on the entire set of sequences.  I showed that UPP typically resulted in better alignments on both DNA and amino acid sequences, which in turn, resulted in more accurate phylogenies compared to other methods.  In addition, I showed that UPP could accurately estimate an alignment on 1,000,000 sequences without the need of a supercomputer in less than 2 days.

\section{Future Work}
SEPP, TIPP, and UPP all use a similar pipeline for sequence alignment: a backbone alignment is decomposed into closely related subalignments using a phylogenetic tree, and the query sequences are aligned to the subalignments.  Any improvement to this pipeline could potentially result in improved accuracy in each of the three techniques.  Possible ways to improve the accuracy of the pipeline include:
  
\begin{itemize}
\item Using different methods for aligning the query sequence to the backbone alignment.  For example, we saw in Chapter~\ref{upp_chapter} that Mafft-profile run under the most accurate settings resulted in accurate alignments on datasets containing both fragmentary and full-length sequences.  However, this setting of Mafft could not be run on the larger datasets.  The subalignments generated by the decomposition process could be made sufficiently small enough such that Mafft-profile could be run optimally.
\item Applying different methods for decomposing the backbone alignment.  In all three methods presented, the backbone tree was decomposed using the centroid edge decomposition.  Using a different technique, such as the longest edge decomposition used in SAT\'{e}~\cite{Liu2009}, may result in better HMMs.  Similarly, using a clade-based decomposition to group together taxonomically similar sequences may result in better HMMs.
\item Using the hierarchal family of HMMs within SEPP and TIPP.  Currently, only UPP has been tested with the hierarchal family of HMMs.  However, the hierarchal family may also lead to better placement results and taxonomic profiling results.  In both SEPP and TIPP, the HMMs are computed on subalignments of roughly the same size.  However, we saw in Chapter~\ref{upp_chapter} that the HMM that yields the best HMMER score is not always the HMM based upon the smallest alignment subsets.  By using the hierarchical family of HMMs, we allow fragments to align to both small and large HMMs which may result in improved alignment accuracy.

%By using the hierarchical family of HMMs, fragments can be aligned to the HMM computed on variable sized subalignments, and thus, is free to select 
%\item Using different parameter settings within the HMMER software suite.  By default, HMMBUILD collapses very gappy sites into one state in the HMM.  However, SEPP, TIPP, and UPP have this option disabled; each site with at least one non-gap character in the backbone alignment results in one state in the HMM.  Better performance may be obtained by allowing gappy sites to be collapsed.
\end{itemize}

Future research for TIPP includes simple changes such as expanding the reference dataset and algorithmic changes such as modification to identification and profiling.  Future work includes:

\begin{itemize}
\item Expanding the marker gene set.  TIPP currently uses a set of 30 marker genes for taxonomic profiling.  A simple extension would be to expand the marker gene set to the 40 marker genes used in~\cite{Sunagawa2013}, as well as update existing marker genes to include recently sequenced species.  By expanding the marker gene set, TIPP may be able to estimate more accurate profiles on metagenomic datasets.  
%\item Better curation of the marker set.  As Table~\ref{tipp:marker_stats} shows, the number of species per marker gene can range from 65 species to 1555 species.  Thus, some markers may 
\item Exploring taxonomic identification of viral sequences.  Viruses are difficult to identify because there are no genes that are found across all viruses.  Instead, viruses are typically identified using group specific genes.  To make the problem more difficult, viruses have high rates of mutations and horizontal gene transfer, making alignment estimation and phylogeny estimation difficult~\cite{todo}.  Thus, viral identification would be a good test case for TIPP's ability to classify divergent sequences.
\item Combining abundance profiles.  TIPP currently uses a simplistic algorithm for computing the taxonomic profile from the marker genes; all reads that are binned to any of the marker genes are pooled together, and the abundance profile is estimated on the pooled reads.  This process ignores the fact that the source gene of the reads is known.  Better profiles may be obtained if separate abundance profiles are estimated from the reads binned to each individual marker gene, and the profiles are combined using a mixture modeling approach.  
\item Improved detection of rare species in a sample.  One difficulty in taxonomic profiling is determining whether a low abundance species is truly present, or the abundance estimation is a false positive.  While TIPP treats each read independently, the reads themselves are not independent; they come from population of species present in the metagenomic sample.  Thus, inferences about the abundance profile of the reads can be used to filter out false positives, as well as detect rare species.  For example, if a rare species is detected across multiple different markers, it's likely to be present in the sample.  However, if the species is only present in very few markers, then it is more likely to be a false positive. 
\end{itemize}

Future work on UPP include:
\begin{itemize}
\item Incorporating iteration within UPP.  The quality of the final alignment can be heavily dependent on the initial selection of the backbone sequences.  The initial step of filtering short sequences from the backbone selection process may exclude entire clades from the backbone set, making it difficult to align sequences from the excluded clade.  I have already shown preliminary work that iteration within UPP can result in better alignments when the initial backbone set is sampled non-uniformly from the phylogenetic tree.  Re-sampling may also be necessary due to random sampling failing to include any sequences from smaller clades.  Better results could be obtained by examining different re-sampling strategies, as well as examining different ways of selecting the backbone set.  
\end{itemize}


%\textbf{Citations to include}

%\textbf{Nam:  rewrite the future work section.  Split into 4 sections:  general improvements that impact all 3 methods, and individual sections for SEPP, TIPP, and UPP sections.}
